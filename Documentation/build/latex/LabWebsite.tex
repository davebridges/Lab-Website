% Generated by Sphinx.
\def\sphinxdocclass{report}
\documentclass[letterpaper,10pt,english]{sphinxmanual}
\usepackage[utf8]{inputenc}
\DeclareUnicodeCharacter{00A0}{\nobreakspace}
\usepackage[T1]{fontenc}
\usepackage{babel}
\usepackage{times}
\usepackage[Bjarne]{fncychap}
\usepackage{longtable}
\usepackage{sphinx}
\usepackage{multirow}


\title{Lab Website Documentation}
\date{September 29, 2012}
\release{0.1.0}
\author{Dave Bridges}
\newcommand{\sphinxlogo}{}
\renewcommand{\releasename}{Release}
\makeindex

\makeatletter
\def\PYG@reset{\let\PYG@it=\relax \let\PYG@bf=\relax%
    \let\PYG@ul=\relax \let\PYG@tc=\relax%
    \let\PYG@bc=\relax \let\PYG@ff=\relax}
\def\PYG@tok#1{\csname PYG@tok@#1\endcsname}
\def\PYG@toks#1+{\ifx\relax#1\empty\else%
    \PYG@tok{#1}\expandafter\PYG@toks\fi}
\def\PYG@do#1{\PYG@bc{\PYG@tc{\PYG@ul{%
    \PYG@it{\PYG@bf{\PYG@ff{#1}}}}}}}
\def\PYG#1#2{\PYG@reset\PYG@toks#1+\relax+\PYG@do{#2}}

\expandafter\def\csname PYG@tok@gd\endcsname{\def\PYG@tc##1{\textcolor[rgb]{0.63,0.00,0.00}{##1}}}
\expandafter\def\csname PYG@tok@gu\endcsname{\let\PYG@bf=\textbf\def\PYG@tc##1{\textcolor[rgb]{0.50,0.00,0.50}{##1}}}
\expandafter\def\csname PYG@tok@gt\endcsname{\def\PYG@tc##1{\textcolor[rgb]{0.00,0.25,0.82}{##1}}}
\expandafter\def\csname PYG@tok@gs\endcsname{\let\PYG@bf=\textbf}
\expandafter\def\csname PYG@tok@gr\endcsname{\def\PYG@tc##1{\textcolor[rgb]{1.00,0.00,0.00}{##1}}}
\expandafter\def\csname PYG@tok@cm\endcsname{\let\PYG@it=\textit\def\PYG@tc##1{\textcolor[rgb]{0.25,0.50,0.56}{##1}}}
\expandafter\def\csname PYG@tok@vg\endcsname{\def\PYG@tc##1{\textcolor[rgb]{0.73,0.38,0.84}{##1}}}
\expandafter\def\csname PYG@tok@m\endcsname{\def\PYG@tc##1{\textcolor[rgb]{0.13,0.50,0.31}{##1}}}
\expandafter\def\csname PYG@tok@mh\endcsname{\def\PYG@tc##1{\textcolor[rgb]{0.13,0.50,0.31}{##1}}}
\expandafter\def\csname PYG@tok@cs\endcsname{\def\PYG@tc##1{\textcolor[rgb]{0.25,0.50,0.56}{##1}}\def\PYG@bc##1{\setlength{\fboxsep}{0pt}\colorbox[rgb]{1.00,0.94,0.94}{\strut ##1}}}
\expandafter\def\csname PYG@tok@ge\endcsname{\let\PYG@it=\textit}
\expandafter\def\csname PYG@tok@vc\endcsname{\def\PYG@tc##1{\textcolor[rgb]{0.73,0.38,0.84}{##1}}}
\expandafter\def\csname PYG@tok@il\endcsname{\def\PYG@tc##1{\textcolor[rgb]{0.13,0.50,0.31}{##1}}}
\expandafter\def\csname PYG@tok@go\endcsname{\def\PYG@tc##1{\textcolor[rgb]{0.19,0.19,0.19}{##1}}}
\expandafter\def\csname PYG@tok@cp\endcsname{\def\PYG@tc##1{\textcolor[rgb]{0.00,0.44,0.13}{##1}}}
\expandafter\def\csname PYG@tok@gi\endcsname{\def\PYG@tc##1{\textcolor[rgb]{0.00,0.63,0.00}{##1}}}
\expandafter\def\csname PYG@tok@gh\endcsname{\let\PYG@bf=\textbf\def\PYG@tc##1{\textcolor[rgb]{0.00,0.00,0.50}{##1}}}
\expandafter\def\csname PYG@tok@ni\endcsname{\let\PYG@bf=\textbf\def\PYG@tc##1{\textcolor[rgb]{0.84,0.33,0.22}{##1}}}
\expandafter\def\csname PYG@tok@nl\endcsname{\let\PYG@bf=\textbf\def\PYG@tc##1{\textcolor[rgb]{0.00,0.13,0.44}{##1}}}
\expandafter\def\csname PYG@tok@nn\endcsname{\let\PYG@bf=\textbf\def\PYG@tc##1{\textcolor[rgb]{0.05,0.52,0.71}{##1}}}
\expandafter\def\csname PYG@tok@no\endcsname{\def\PYG@tc##1{\textcolor[rgb]{0.38,0.68,0.84}{##1}}}
\expandafter\def\csname PYG@tok@na\endcsname{\def\PYG@tc##1{\textcolor[rgb]{0.25,0.44,0.63}{##1}}}
\expandafter\def\csname PYG@tok@nb\endcsname{\def\PYG@tc##1{\textcolor[rgb]{0.00,0.44,0.13}{##1}}}
\expandafter\def\csname PYG@tok@nc\endcsname{\let\PYG@bf=\textbf\def\PYG@tc##1{\textcolor[rgb]{0.05,0.52,0.71}{##1}}}
\expandafter\def\csname PYG@tok@nd\endcsname{\let\PYG@bf=\textbf\def\PYG@tc##1{\textcolor[rgb]{0.33,0.33,0.33}{##1}}}
\expandafter\def\csname PYG@tok@ne\endcsname{\def\PYG@tc##1{\textcolor[rgb]{0.00,0.44,0.13}{##1}}}
\expandafter\def\csname PYG@tok@nf\endcsname{\def\PYG@tc##1{\textcolor[rgb]{0.02,0.16,0.49}{##1}}}
\expandafter\def\csname PYG@tok@si\endcsname{\let\PYG@it=\textit\def\PYG@tc##1{\textcolor[rgb]{0.44,0.63,0.82}{##1}}}
\expandafter\def\csname PYG@tok@s2\endcsname{\def\PYG@tc##1{\textcolor[rgb]{0.25,0.44,0.63}{##1}}}
\expandafter\def\csname PYG@tok@vi\endcsname{\def\PYG@tc##1{\textcolor[rgb]{0.73,0.38,0.84}{##1}}}
\expandafter\def\csname PYG@tok@nt\endcsname{\let\PYG@bf=\textbf\def\PYG@tc##1{\textcolor[rgb]{0.02,0.16,0.45}{##1}}}
\expandafter\def\csname PYG@tok@nv\endcsname{\def\PYG@tc##1{\textcolor[rgb]{0.73,0.38,0.84}{##1}}}
\expandafter\def\csname PYG@tok@s1\endcsname{\def\PYG@tc##1{\textcolor[rgb]{0.25,0.44,0.63}{##1}}}
\expandafter\def\csname PYG@tok@gp\endcsname{\let\PYG@bf=\textbf\def\PYG@tc##1{\textcolor[rgb]{0.78,0.36,0.04}{##1}}}
\expandafter\def\csname PYG@tok@sh\endcsname{\def\PYG@tc##1{\textcolor[rgb]{0.25,0.44,0.63}{##1}}}
\expandafter\def\csname PYG@tok@ow\endcsname{\let\PYG@bf=\textbf\def\PYG@tc##1{\textcolor[rgb]{0.00,0.44,0.13}{##1}}}
\expandafter\def\csname PYG@tok@sx\endcsname{\def\PYG@tc##1{\textcolor[rgb]{0.78,0.36,0.04}{##1}}}
\expandafter\def\csname PYG@tok@bp\endcsname{\def\PYG@tc##1{\textcolor[rgb]{0.00,0.44,0.13}{##1}}}
\expandafter\def\csname PYG@tok@c1\endcsname{\let\PYG@it=\textit\def\PYG@tc##1{\textcolor[rgb]{0.25,0.50,0.56}{##1}}}
\expandafter\def\csname PYG@tok@kc\endcsname{\let\PYG@bf=\textbf\def\PYG@tc##1{\textcolor[rgb]{0.00,0.44,0.13}{##1}}}
\expandafter\def\csname PYG@tok@c\endcsname{\let\PYG@it=\textit\def\PYG@tc##1{\textcolor[rgb]{0.25,0.50,0.56}{##1}}}
\expandafter\def\csname PYG@tok@mf\endcsname{\def\PYG@tc##1{\textcolor[rgb]{0.13,0.50,0.31}{##1}}}
\expandafter\def\csname PYG@tok@err\endcsname{\def\PYG@bc##1{\setlength{\fboxsep}{0pt}\fcolorbox[rgb]{1.00,0.00,0.00}{1,1,1}{\strut ##1}}}
\expandafter\def\csname PYG@tok@kd\endcsname{\let\PYG@bf=\textbf\def\PYG@tc##1{\textcolor[rgb]{0.00,0.44,0.13}{##1}}}
\expandafter\def\csname PYG@tok@ss\endcsname{\def\PYG@tc##1{\textcolor[rgb]{0.32,0.47,0.09}{##1}}}
\expandafter\def\csname PYG@tok@sr\endcsname{\def\PYG@tc##1{\textcolor[rgb]{0.14,0.33,0.53}{##1}}}
\expandafter\def\csname PYG@tok@mo\endcsname{\def\PYG@tc##1{\textcolor[rgb]{0.13,0.50,0.31}{##1}}}
\expandafter\def\csname PYG@tok@mi\endcsname{\def\PYG@tc##1{\textcolor[rgb]{0.13,0.50,0.31}{##1}}}
\expandafter\def\csname PYG@tok@kn\endcsname{\let\PYG@bf=\textbf\def\PYG@tc##1{\textcolor[rgb]{0.00,0.44,0.13}{##1}}}
\expandafter\def\csname PYG@tok@o\endcsname{\def\PYG@tc##1{\textcolor[rgb]{0.40,0.40,0.40}{##1}}}
\expandafter\def\csname PYG@tok@kr\endcsname{\let\PYG@bf=\textbf\def\PYG@tc##1{\textcolor[rgb]{0.00,0.44,0.13}{##1}}}
\expandafter\def\csname PYG@tok@s\endcsname{\def\PYG@tc##1{\textcolor[rgb]{0.25,0.44,0.63}{##1}}}
\expandafter\def\csname PYG@tok@kp\endcsname{\def\PYG@tc##1{\textcolor[rgb]{0.00,0.44,0.13}{##1}}}
\expandafter\def\csname PYG@tok@w\endcsname{\def\PYG@tc##1{\textcolor[rgb]{0.73,0.73,0.73}{##1}}}
\expandafter\def\csname PYG@tok@kt\endcsname{\def\PYG@tc##1{\textcolor[rgb]{0.56,0.13,0.00}{##1}}}
\expandafter\def\csname PYG@tok@sc\endcsname{\def\PYG@tc##1{\textcolor[rgb]{0.25,0.44,0.63}{##1}}}
\expandafter\def\csname PYG@tok@sb\endcsname{\def\PYG@tc##1{\textcolor[rgb]{0.25,0.44,0.63}{##1}}}
\expandafter\def\csname PYG@tok@k\endcsname{\let\PYG@bf=\textbf\def\PYG@tc##1{\textcolor[rgb]{0.00,0.44,0.13}{##1}}}
\expandafter\def\csname PYG@tok@se\endcsname{\let\PYG@bf=\textbf\def\PYG@tc##1{\textcolor[rgb]{0.25,0.44,0.63}{##1}}}
\expandafter\def\csname PYG@tok@sd\endcsname{\let\PYG@it=\textit\def\PYG@tc##1{\textcolor[rgb]{0.25,0.44,0.63}{##1}}}

\def\PYGZbs{\char`\\}
\def\PYGZus{\char`\_}
\def\PYGZob{\char`\{}
\def\PYGZcb{\char`\}}
\def\PYGZca{\char`\^}
\def\PYGZam{\char`\&}
\def\PYGZlt{\char`\<}
\def\PYGZgt{\char`\>}
\def\PYGZsh{\char`\#}
\def\PYGZpc{\char`\%}
\def\PYGZdl{\char`\$}
\def\PYGZti{\char`\~}
% for compatibility with earlier versions
\def\PYGZat{@}
\def\PYGZlb{[}
\def\PYGZrb{]}
\makeatother

\begin{document}

\maketitle
\tableofcontents
\phantomsection\label{index::doc}


Contents:


\chapter{Papers}
\label{papers:papers}\label{papers::doc}\label{papers:module-papers}\label{papers:welcome-to-lab-website-s-documentation}\index{papers (module)}
This application will store and display the relevant {\hyperref[papers:papers.models.Publication]{\code{Publication}}} objects and associated data.


\section{Current Functionality}
\label{papers:current-functionality}\begin{itemize}
\item {} 
Includes and displays papers from the lab or other interesting papers.

\item {} 
Papers are marked up with microdata markup from \href{http://schema.org}{http://schema.org}.

\item {} 
There is an API estabished which serves some {\hyperref[papers:papers.models.Publication]{\code{Publication}}} information in json or xml format.  See {\hyperref[papers:module-papers.api]{\code{papers.api}}} for details.

\item {} 
There will be included comment threads at each paper served by Disqus.

\item {} 
API's from Altmetric and PLOS are used to also display altmetrics for these papers.

\item {} 
Papers will also link to author profiles (see the \code{personnel} app).

\item {} 
It is possible to tweet, like (through facebook) or +1 (through google plus) a paper, though the functionality of these are not yet refined.

\end{itemize}


\section{Longer Term Goals}
\label{papers:longer-term-goals}\begin{itemize}
\item {} 
Another idea is to have hidden discussions of papers, or papers which are not publicly displayed and require a login.

\item {} 
Incorporate Mendeley, TotalImpact and PubMedCentral APIs.

\item {} 
Potentially convert to a facebook app with custom actions and objects.

\item {} 
Markup with opengraph tags and incorporate twitter cards.

\item {} 
The {\hyperref[papers:papers.models.Publication]{\code{Publication}}} objects are manually entered but I hope to have these be automatically be generated from CrossRef or Mendeley APIs

\end{itemize}


\section{Source Code Documentation}
\label{papers:source-code-documentation}

\subsection{Models}
\label{papers:module-papers.models}\label{papers:models}\index{papers.models (module)}
This file is the model configuration file for the :mod{}`papers{}` app.

There are two models in this app, {\hyperref[papers:papers.models.Publication]{\code{Publication}}} and {\hyperref[papers:papers.models.AuthorDetails]{\code{AuthorDetails}}}
\index{AuthorDetails (class in papers.models)}

\begin{fulllineitems}
\phantomsection\label{papers:papers.models.AuthorDetails}\pysiglinewithargsret{\strong{class }\code{papers.models.}\bfcode{AuthorDetails}}{\emph{*args}, \emph{**kwargs}}{}
This is a group of authors for a specific paper.

Because each {\hyperref[papers:papers.models.Publication]{\code{Publication}}} has a list of authors and the order matters, the authors are listed in this linked model.
The authors are defined by the \code{Person} model class, which is also the UserProfile class.
This model has a ManyToMany link with a paper as well as marks for order, and whether an author is a corresponding or equally contributing author.

\end{fulllineitems}

\index{Publication (class in papers.models)}

\begin{fulllineitems}
\phantomsection\label{papers:papers.models.Publication}\pysiglinewithargsret{\strong{class }\code{papers.models.}\bfcode{Publication}}{\emph{*args}, \emph{**kwargs}}{}
This model covers {\hyperref[papers:papers.models.Publication]{\code{Publication}}} objects of several types.

The publication fields are based on Mendeley and PubMed fields.
For the author, there is a ManyToMany link to a group of authors with the order and other details, see {\hyperref[papers:papers.models.AuthorDetails]{\code{AuthorDetails}}}.
\index{doi\_link() (papers.models.Publication method)}

\begin{fulllineitems}
\phantomsection\label{papers:papers.models.Publication.doi_link}\pysiglinewithargsret{\bfcode{doi\_link}}{}{}
This turns the DOI into a link.

\end{fulllineitems}

\index{full\_pmcid() (papers.models.Publication method)}

\begin{fulllineitems}
\phantomsection\label{papers:papers.models.Publication.full_pmcid}\pysiglinewithargsret{\bfcode{full\_pmcid}}{}{}
Converts the integer to a full PMCID

\end{fulllineitems}

\index{get\_absolute\_url() (papers.models.Publication method)}

\begin{fulllineitems}
\phantomsection\label{papers:papers.models.Publication.get_absolute_url}\pysiglinewithargsret{\bfcode{get\_absolute\_url}}{\emph{*moreargs}, \emph{**morekwargs}}{}
the permalink for a paper detail page is \textbf{/papers/\textless{}title\_slug\textgreater{}}

\end{fulllineitems}

\index{save() (papers.models.Publication method)}

\begin{fulllineitems}
\phantomsection\label{papers:papers.models.Publication.save}\pysiglinewithargsret{\bfcode{save}}{\emph{*args}, \emph{**kwargs}}{}
The title is slugified upon saving into title\_slug.

\end{fulllineitems}


\end{fulllineitems}



\subsection{Views}
\label{papers:module-papers.views}\label{papers:views}\index{papers.views (module)}
This app contains the views for the :mod{}`papers{}` app.

There are three views for this app, {\hyperref[papers:papers.views.LaboratoryPaperList]{\code{LaboratoryPaperList}}}, {\hyperref[papers:papers.views.InterestingPaperList]{\code{InterestingPaperList}}} and {\hyperref[papers:papers.views.PaperDetailView]{\code{PaperDetailView}}}
\index{InterestingPaperList (class in papers.views)}

\begin{fulllineitems}
\phantomsection\label{papers:papers.views.InterestingPaperList}\pysiglinewithargsret{\strong{class }\code{papers.views.}\bfcode{InterestingPaperList}}{\emph{**kwargs}}{}
This class generates the view for interesting-papers located at \textbf{/papers/interesting}.

This is filtered based on whether the {\hyperref[papers:papers.models.Publication]{\code{Publication}}} is marked as interesting\_paper = True.
\index{get\_context\_data() (papers.views.InterestingPaperList method)}

\begin{fulllineitems}
\phantomsection\label{papers:papers.views.InterestingPaperList.get_context_data}\pysiglinewithargsret{\bfcode{get\_context\_data}}{\emph{**kwargs}}{}
This method adds to the context the paper-list-type  = interesting.

\end{fulllineitems}


\end{fulllineitems}

\index{LaboratoryPaperList (class in papers.views)}

\begin{fulllineitems}
\phantomsection\label{papers:papers.views.LaboratoryPaperList}\pysiglinewithargsret{\strong{class }\code{papers.views.}\bfcode{LaboratoryPaperList}}{\emph{**kwargs}}{}
This class generates the view for laboratory-papers located at \textbf{/papers}.

This is filtered based on whether the {\hyperref[papers:papers.models.Publication]{\code{Publication}}} is marked as laboratory\_paper = True.
\index{get\_context\_data() (papers.views.LaboratoryPaperList method)}

\begin{fulllineitems}
\phantomsection\label{papers:papers.views.LaboratoryPaperList.get_context_data}\pysiglinewithargsret{\bfcode{get\_context\_data}}{\emph{**kwargs}}{}
This method adds to the context the paper-list-type  = interesting.

\end{fulllineitems}


\end{fulllineitems}

\index{PaperCreate (class in papers.views)}

\begin{fulllineitems}
\phantomsection\label{papers:papers.views.PaperCreate}\pysiglinewithargsret{\strong{class }\code{papers.views.}\bfcode{PaperCreate}}{\emph{**kwargs}}{}
This view is for creating a new {\hyperref[papers:papers.models.Publication]{\code{Publication}}}.

It requires the permissions to create a new paper and is found at the url \textbf{/paper/new}.
\index{model (papers.views.PaperCreate attribute)}

\begin{fulllineitems}
\phantomsection\label{papers:papers.views.PaperCreate.model}\pysigline{\bfcode{model}}
alias of \code{Publication}

\end{fulllineitems}


\end{fulllineitems}

\index{PaperDelete (class in papers.views)}

\begin{fulllineitems}
\phantomsection\label{papers:papers.views.PaperDelete}\pysiglinewithargsret{\strong{class }\code{papers.views.}\bfcode{PaperDelete}}{\emph{**kwargs}}{}
This view is for deleting a {\hyperref[papers:papers.models.Publication]{\code{Publication}}}.

It requires the permissions to delete a paper and is found at the url \textbf{/paper/\textless{}slug\textgreater{}/delete}.
\index{model (papers.views.PaperDelete attribute)}

\begin{fulllineitems}
\phantomsection\label{papers:papers.views.PaperDelete.model}\pysigline{\bfcode{model}}
alias of \code{Publication}

\end{fulllineitems}


\end{fulllineitems}

\index{PaperDetailView (class in papers.views)}

\begin{fulllineitems}
\phantomsection\label{papers:papers.views.PaperDetailView}\pysiglinewithargsret{\strong{class }\code{papers.views.}\bfcode{PaperDetailView}}{\emph{**kwargs}}{}
This class generates the view for paper-details located at \textbf{/papers/\textless{}title\_slug\textgreater{}}.
\index{model (papers.views.PaperDetailView attribute)}

\begin{fulllineitems}
\phantomsection\label{papers:papers.views.PaperDetailView.model}\pysigline{\bfcode{model}}
alias of \code{Publication}

\end{fulllineitems}

\index{render\_to\_response() (papers.views.PaperDetailView method)}

\begin{fulllineitems}
\phantomsection\label{papers:papers.views.PaperDetailView.render_to_response}\pysiglinewithargsret{\bfcode{render\_to\_response}}{\emph{context}, \emph{**kwargs}}{}
The render\_to\_response for this view is over-ridden to add the api\_keys context processor.

\end{fulllineitems}


\end{fulllineitems}

\index{PaperUpdate (class in papers.views)}

\begin{fulllineitems}
\phantomsection\label{papers:papers.views.PaperUpdate}\pysiglinewithargsret{\strong{class }\code{papers.views.}\bfcode{PaperUpdate}}{\emph{**kwargs}}{}
This view is for updating a {\hyperref[papers:papers.models.Publication]{\code{Publication}}}.

It requires the permissions to update a paper and is found at the url \textbf{/paper/\textless{}slug\textgreater{}/edit}.
\index{model (papers.views.PaperUpdate attribute)}

\begin{fulllineitems}
\phantomsection\label{papers:papers.views.PaperUpdate.model}\pysigline{\bfcode{model}}
alias of \code{Publication}

\end{fulllineitems}


\end{fulllineitems}



\subsection{URLS}
\label{papers:module-papers.urls}\label{papers:urls}\index{papers.urls (module)}
This package has the url encodings for the {\hyperref[papers:module-papers]{\code{papers}}} app.


\subsection{Tests}
\label{papers:module-papers.tests}\label{papers:tests}\index{papers.tests (module)}
This package contains the unit tests for the {\hyperref[papers:module-papers]{\code{papers}}} app.

It contains view and model tests for each model, grouped together.
Contains the two model tests:
\begin{itemize}
\item {} 
{\hyperref[papers:papers.tests.PublicationModelTests]{\code{PublicationModelTests}}}

\item {} 
{\hyperref[papers:papers.tests.AuthorDetailsModelTests]{\code{AuthorDetailsModelTests}}}

\end{itemize}

The API tests:
\begin{itemize}
\item {} 
{\hyperref[papers:papers.tests.PublicationResourceTests]{\code{PublicationResourceTests}}}

\end{itemize}

And the view tests:
\begin{itemize}
\item {} 
{\hyperref[papers:papers.tests.PublicationViewTests]{\code{PublicationViewTests}}}

\end{itemize}
\index{AuthorDetailsModelTests (class in papers.tests)}

\begin{fulllineitems}
\phantomsection\label{papers:papers.tests.AuthorDetailsModelTests}\pysiglinewithargsret{\strong{class }\code{papers.tests.}\bfcode{AuthorDetailsModelTests}}{\emph{methodName='runTest'}}{}
This class tests varios aspects of the {\hyperref[papers:papers.models.AuthorDetails]{\code{AuthorDetails}}} model.
\index{setUp() (papers.tests.AuthorDetailsModelTests method)}

\begin{fulllineitems}
\phantomsection\label{papers:papers.tests.AuthorDetailsModelTests.setUp}\pysiglinewithargsret{\bfcode{setUp}}{}{}
Instantiate the test client.  Creates a test user.

\end{fulllineitems}

\index{tearDown() (papers.tests.AuthorDetailsModelTests method)}

\begin{fulllineitems}
\phantomsection\label{papers:papers.tests.AuthorDetailsModelTests.tearDown}\pysiglinewithargsret{\bfcode{tearDown}}{}{}
Depopulate created model instances from test database.

\end{fulllineitems}

\index{test\_authordetail\_unicode() (papers.tests.AuthorDetailsModelTests method)}

\begin{fulllineitems}
\phantomsection\label{papers:papers.tests.AuthorDetailsModelTests.test_authordetail_unicode}\pysiglinewithargsret{\bfcode{test\_authordetail\_unicode}}{}{}
This tests that the unicode representaton of an {\hyperref[papers:papers.models.AuthorDetails]{\code{AuthorDetails}}} object is correct.

\end{fulllineitems}

\index{test\_create\_new\_authordetail\_all() (papers.tests.AuthorDetailsModelTests method)}

\begin{fulllineitems}
\phantomsection\label{papers:papers.tests.AuthorDetailsModelTests.test_create_new_authordetail_all}\pysiglinewithargsret{\bfcode{test\_create\_new\_authordetail\_all}}{}{}
This test creates a {\hyperref[papers:papers.models.AuthorDetails]{\code{AuthorDetails}}} with the required information only.

\end{fulllineitems}

\index{test\_create\_new\_authordetail\_minimum() (papers.tests.AuthorDetailsModelTests method)}

\begin{fulllineitems}
\phantomsection\label{papers:papers.tests.AuthorDetailsModelTests.test_create_new_authordetail_minimum}\pysiglinewithargsret{\bfcode{test\_create\_new\_authordetail\_minimum}}{}{}
This test creates a {\hyperref[papers:papers.models.AuthorDetails]{\code{AuthorDetails}}} with the required information only.

\end{fulllineitems}


\end{fulllineitems}

\index{PublicationModelTests (class in papers.tests)}

\begin{fulllineitems}
\phantomsection\label{papers:papers.tests.PublicationModelTests}\pysiglinewithargsret{\strong{class }\code{papers.tests.}\bfcode{PublicationModelTests}}{\emph{methodName='runTest'}}{}
This class tests various aspects of the {\hyperref[papers:papers.models.Publication]{\code{Publication}}} model.
\index{setUp() (papers.tests.PublicationModelTests method)}

\begin{fulllineitems}
\phantomsection\label{papers:papers.tests.PublicationModelTests.setUp}\pysiglinewithargsret{\bfcode{setUp}}{}{}
Instantiate the test client.  Creates a test user.

\end{fulllineitems}

\index{tearDown() (papers.tests.PublicationModelTests method)}

\begin{fulllineitems}
\phantomsection\label{papers:papers.tests.PublicationModelTests.tearDown}\pysiglinewithargsret{\bfcode{tearDown}}{}{}
Depopulate created model instances from test database.

\end{fulllineitems}

\index{test\_create\_new\_paper\_minimum() (papers.tests.PublicationModelTests method)}

\begin{fulllineitems}
\phantomsection\label{papers:papers.tests.PublicationModelTests.test_create_new_paper_minimum}\pysiglinewithargsret{\bfcode{test\_create\_new\_paper\_minimum}}{}{}
This test creates a {\hyperref[papers:papers.models.Publication]{\code{Publication}}} with the required information only.

\end{fulllineitems}

\index{test\_full\_pmcid() (papers.tests.PublicationModelTests method)}

\begin{fulllineitems}
\phantomsection\label{papers:papers.tests.PublicationModelTests.test_full_pmcid}\pysiglinewithargsret{\bfcode{test\_full\_pmcid}}{}{}
This tests that a correct full PMCID can be generated for a {\hyperref[papers:papers.models.Publication]{\code{Publication}}}.

\end{fulllineitems}

\index{test\_paper\_absolute\_url() (papers.tests.PublicationModelTests method)}

\begin{fulllineitems}
\phantomsection\label{papers:papers.tests.PublicationModelTests.test_paper_absolute_url}\pysiglinewithargsret{\bfcode{test\_paper\_absolute\_url}}{}{}
This tests the title\_slug field of a {\hyperref[papers:papers.models.Publication]{\code{Publication}}}.

\end{fulllineitems}

\index{test\_paper\_doi\_link() (papers.tests.PublicationModelTests method)}

\begin{fulllineitems}
\phantomsection\label{papers:papers.tests.PublicationModelTests.test_paper_doi_link}\pysiglinewithargsret{\bfcode{test\_paper\_doi\_link}}{}{}
This tests the title\_slug field of a {\hyperref[papers:papers.models.Publication]{\code{Publication}}}.

\end{fulllineitems}

\index{test\_paper\_title\_slug() (papers.tests.PublicationModelTests method)}

\begin{fulllineitems}
\phantomsection\label{papers:papers.tests.PublicationModelTests.test_paper_title_slug}\pysiglinewithargsret{\bfcode{test\_paper\_title\_slug}}{}{}
This tests the title\_slug field of a {\hyperref[papers:papers.models.Publication]{\code{Publication}}}.

\end{fulllineitems}

\index{test\_paper\_unicode() (papers.tests.PublicationModelTests method)}

\begin{fulllineitems}
\phantomsection\label{papers:papers.tests.PublicationModelTests.test_paper_unicode}\pysiglinewithargsret{\bfcode{test\_paper\_unicode}}{}{}
This tests the unicode representation of a {\hyperref[papers:papers.models.Publication]{\code{Publication}}}.

\end{fulllineitems}


\end{fulllineitems}

\index{PublicationResourceTests (class in papers.tests)}

\begin{fulllineitems}
\phantomsection\label{papers:papers.tests.PublicationResourceTests}\pysiglinewithargsret{\strong{class }\code{papers.tests.}\bfcode{PublicationResourceTests}}{\emph{methodName='runTest'}}{}
This class tests varios aspects of the {\hyperref[papers:papers.api.PublicationResource]{\code{PublicationResource}}} API model.
\index{api\_publication\_detail\_test() (papers.tests.PublicationResourceTests method)}

\begin{fulllineitems}
\phantomsection\label{papers:papers.tests.PublicationResourceTests.api_publication_detail_test}\pysiglinewithargsret{\bfcode{api\_publication\_detail\_test}}{}{}
This tests that the API correctly renders a particular {\hyperref[papers:papers.models.Publication]{\code{Publication}}} objects.

\end{fulllineitems}

\index{api\_publication\_list\_test() (papers.tests.PublicationResourceTests method)}

\begin{fulllineitems}
\phantomsection\label{papers:papers.tests.PublicationResourceTests.api_publication_list_test}\pysiglinewithargsret{\bfcode{api\_publication\_list\_test}}{}{}
This tests that the API correctly renders a list of {\hyperref[papers:papers.models.Publication]{\code{Publication}}} objects.

\end{fulllineitems}

\index{setUp() (papers.tests.PublicationResourceTests method)}

\begin{fulllineitems}
\phantomsection\label{papers:papers.tests.PublicationResourceTests.setUp}\pysiglinewithargsret{\bfcode{setUp}}{}{}
Instantiate the test client.  Creates a test user.

\end{fulllineitems}

\index{tearDown() (papers.tests.PublicationResourceTests method)}

\begin{fulllineitems}
\phantomsection\label{papers:papers.tests.PublicationResourceTests.tearDown}\pysiglinewithargsret{\bfcode{tearDown}}{}{}
Depopulate created model instances from test database.

\end{fulllineitems}


\end{fulllineitems}

\index{PublicationViewTests (class in papers.tests)}

\begin{fulllineitems}
\phantomsection\label{papers:papers.tests.PublicationViewTests}\pysiglinewithargsret{\strong{class }\code{papers.tests.}\bfcode{PublicationViewTests}}{\emph{methodName='runTest'}}{}
This class tests the views for {\hyperref[papers:papers.models.Publication]{\code{Publication}}} objects.
\index{setUp() (papers.tests.PublicationViewTests method)}

\begin{fulllineitems}
\phantomsection\label{papers:papers.tests.PublicationViewTests.setUp}\pysiglinewithargsret{\bfcode{setUp}}{}{}
Instantiate the test client.  Creates a test user.

\end{fulllineitems}

\index{tearDown() (papers.tests.PublicationViewTests method)}

\begin{fulllineitems}
\phantomsection\label{papers:papers.tests.PublicationViewTests.tearDown}\pysiglinewithargsret{\bfcode{tearDown}}{}{}
Depopulate created model instances from test database.

\end{fulllineitems}

\index{test\_interesting\_papers\_list() (papers.tests.PublicationViewTests method)}

\begin{fulllineitems}
\phantomsection\label{papers:papers.tests.PublicationViewTests.test_interesting_papers_list}\pysiglinewithargsret{\bfcode{test\_interesting\_papers\_list}}{}{}
This tests the interesting-papers view ensuring that templates are loaded correctly.

This view uses a user with superuser permissions so does not test the permission levels for this view.

\end{fulllineitems}

\index{test\_lab\_papers\_list() (papers.tests.PublicationViewTests method)}

\begin{fulllineitems}
\phantomsection\label{papers:papers.tests.PublicationViewTests.test_lab_papers_list}\pysiglinewithargsret{\bfcode{test\_lab\_papers\_list}}{}{}
This tests the laboratory-papers view ensuring that templates are loaded correctly.

This view uses a user with superuser permissions so does not test the permission levels for this view.

\end{fulllineitems}

\index{test\_publication\_view() (papers.tests.PublicationViewTests method)}

\begin{fulllineitems}
\phantomsection\label{papers:papers.tests.PublicationViewTests.test_publication_view}\pysiglinewithargsret{\bfcode{test\_publication\_view}}{}{}
This tests the paper-details view, ensuring that templates are loaded correctly.

This view uses a user with superuser permissions so does not test the permission levels for this view.

\end{fulllineitems}

\index{test\_publication\_view\_create() (papers.tests.PublicationViewTests method)}

\begin{fulllineitems}
\phantomsection\label{papers:papers.tests.PublicationViewTests.test_publication_view_create}\pysiglinewithargsret{\bfcode{test\_publication\_view\_create}}{}{}
This tests the paper-new view, ensuring that templates are loaded correctly.

This view uses a user with superuser permissions so does not test the permission levels for this view.

\end{fulllineitems}

\index{test\_publication\_view\_delete() (papers.tests.PublicationViewTests method)}

\begin{fulllineitems}
\phantomsection\label{papers:papers.tests.PublicationViewTests.test_publication_view_delete}\pysiglinewithargsret{\bfcode{test\_publication\_view\_delete}}{}{}
This tests the paper-delete view, ensuring that templates are loaded correctly.

This view uses a user with superuser permissions so does not test the permission levels for this view.

\end{fulllineitems}

\index{test\_publication\_view\_edit() (papers.tests.PublicationViewTests method)}

\begin{fulllineitems}
\phantomsection\label{papers:papers.tests.PublicationViewTests.test_publication_view_edit}\pysiglinewithargsret{\bfcode{test\_publication\_view\_edit}}{}{}
This tests the paper-edit view, ensuring that templates are loaded correctly.

This view uses a user with superuser permissions so does not test the permission levels for this view.

\end{fulllineitems}


\end{fulllineitems}



\subsection{Admin}
\label{papers:module-papers.admin}\label{papers:admin}\index{papers.admin (module)}
This package sets up the admin interface for the {\hyperref[papers:module-papers]{\code{papers}}} app.
\index{AuthorDetailsAdmin (class in papers.admin)}

\begin{fulllineitems}
\phantomsection\label{papers:papers.admin.AuthorDetailsAdmin}\pysiglinewithargsret{\strong{class }\code{papers.admin.}\bfcode{AuthorDetailsAdmin}}{\emph{model}, \emph{admin\_site}}{}
The {\hyperref[papers:papers.models.AuthorDetails]{\code{AuthorDetails}}} model admin is the default.

\end{fulllineitems}

\index{PublicationAdmin (class in papers.admin)}

\begin{fulllineitems}
\phantomsection\label{papers:papers.admin.PublicationAdmin}\pysiglinewithargsret{\strong{class }\code{papers.admin.}\bfcode{PublicationAdmin}}{\emph{model}, \emph{admin\_site}}{}
The {\hyperref[papers:papers.models.Publication]{\code{Publication}}} model admin is the default.

\end{fulllineitems}



\subsection{API}
\label{papers:api}\label{papers:module-papers.api}\index{papers.api (module)}
This package controls API access to the {\hyperref[papers:module-papers]{\code{papers}}} app.


\subsubsection{Overview}
\label{papers:overview}
The API for the {\hyperref[papers:module-papers]{\code{papers}}} application provides data on publications.  The data can be provided as either a group of publications or as a single publication.  Only GET requests are accepted.
These urls are served at the endpoint \textbf{/api/v1/publications/}, and depends on your server url.  For these examples we will presume that you can reach this endpoint at \textbf{http://yourserver.org/api/v1/publications/}.  Currently for all requests, no authentication is required.  The entire API schema is available at:

\begin{Verbatim}[commandchars=\\\{\}]
http://yourserver.org/api/v1/publications/schema/?format=xml
http://yourserver.org/api/v1/publications/schema/?format=json
\end{Verbatim}


\subsubsection{Sample Code}
\label{papers:sample-code}
Either group requests or single publication requests can be served depending on if the primary key is provided.  The request URI has several parts including the servername, the api version (currently v1) then the item type (publications).  There must be a trailing slash before the request parameters (which are after a \textbf{?} sign and separated by a \textbf{\&} sign).


\paragraph{For a collection of publications}
\label{papers:for-a-collection-of-publications}
For a collection of publications you can request:

\begin{Verbatim}[commandchars=\\\{\}]
http://yourserver.org/api/v1/publications/?format=json
\end{Verbatim}

This would return all publications in the database.  This would return the following json response with two JSON objects, meta and objects.
The meta object contains fields for the limit, next, offset, previous and total\_count for the series of objects requested.  The objects portion is an array of the returned publications.  Note the id field of a publication.  This is used for retrieving a single publication.  Collections can also be filtered based on type or year:

\begin{Verbatim}[commandchars=\\\{\}]
http://yourserver.org/api/v1/publications/?format=json\&year=2012     
http://yourserver.org/api/v1/publications/?format=json\&type=journal-article 
http://yourserver.org/api/v1/publications/?format=json\&type=journal-article\&year=2012
http://yourserver.org/api/v1/publications/set/1;3/?format=json
\end{Verbatim}

The last example requests the publications with id numbers 1 and 3.


\paragraph{For a single publication}
\label{papers:for-a-single-publication}
To retrieve a single publication you need to know the primary key of the object.  This can be found from the id parameter of a collection (see above) or from the actual object page.  You can retrieve details about a single article with a call such as:

\begin{Verbatim}[commandchars=\\\{\}]
http://yourserver.org/api/v1/publications/2/?format=json
\end{Verbatim}

In this case \textbf{2} is the primary key (or id field) of the publication in question.


\subsubsection{Reference}
\label{papers:reference}

\paragraph{Request Parameters}
\label{papers:request-parameters}
The following are the potential request variables.  You must supply a format, but can also filter based on other parameters.  By default 20 items are returned but you can increase this to all by setting limit=0.

\begin{tabulary}{\linewidth}{|L|L|}
\hline
\textbf{
Parameter
} & \textbf{
Potential Values
}\\\hline

format
 & 
\textbf{json} or \textbf{xml}
\\\hline

year
 & 
\textbf{2008}
\\\hline

type
 & 
\textbf{journal-article} or \textbf{book-section}
\\\hline

laboratory\_paper
 & 
\textbf{true} or \textbf{false}
\\\hline

limit
 & 
\textbf{0} for all, any other number
\\\hline
\end{tabulary}



\paragraph{Response Values}
\label{papers:response-values}
The response (in either json or xml) provides the following fields for each object (or for the only object in the case of a single object request).

\begin{tabulary}{\linewidth}{|L|L|L|}
\hline
\textbf{
Field
} & \textbf{
Explanation
} & \textbf{
Sample Value
}\\\hline

absolute\_url
 & 
the url of the page on this site
 & 
/papers/tc10-is-regulated-by-caveolin-in-3t3-l1-adipocytes/
\\\hline

abstract
 & 
abstract or summary
 & 
some text...
\\\hline

date\_added
 & 
data added to this database
 & 
2012-08-18
\\\hline

date\_last\_modified
 & 
last modified in database
 & 
2012-08-25
\\\hline

doi
 & 
digital object identifier
 & 
10.1371/journal.pone.0042451
\\\hline

id
 & 
the database id number
 & 
1
\\\hline

interesting\_paper
 & 
whether the paper is marked as an interesting paper
 & 
false
\\\hline

issue
 & 
the issue of the journal
 & 
8
\\\hline

journal
 & 
the name of the journal
 & 
PLOS One
\\\hline

laboratory\_paper
 & 
whether the paper is from this laboratory
 & 
true
\\\hline

mendeley\_id
 & 
the mendeley id number for the paper
 & 
null
\\\hline

mendeley\_url
 & 
the mendeley url for the paper
 & \\\hline

pages
 & 
page range for the paper
 & 
e42451
\\\hline

pmcid
 & 
PubMed Central id number
 & 
null
\\\hline

pmid
 & 
PubMed id number
 & 
22900022
\\\hline

resource\_uri
 & 
a link to the api for this publication
 & 
/api/v1/publications/1/
\\\hline

title
 & 
the title of the paper
 & 
TC10 Is Regulated by Caveolin in 3T3-L1 Adipocytes.
\\\hline

title\_slug
 & 
slugified title of the paper
 & 
tc10-is-regulated-by-caveolin-in-3t3-l1-adipocytes
\\\hline

type
 & 
type of publication
 & 
journal-article
\\\hline

volume
 & 
volume of the article in a journal
 & 
7
\\\hline

year
 & 
publication year
 & 
2012
\\\hline
\end{tabulary}

\index{PublicationResource (class in papers.api)}

\begin{fulllineitems}
\phantomsection\label{papers:papers.api.PublicationResource}\pysiglinewithargsret{\strong{class }\code{papers.api.}\bfcode{PublicationResource}}{\emph{api\_name=None}}{}
This generates the API resource for {\hyperref[papers:papers.models.Publication]{\code{Publication}}} objects.

It returns all publications in the database.
Authors are currently not linked, as that would require an API to the \code{personnel} app.
\index{PublicationResource.Meta (class in papers.api)}

\begin{fulllineitems}
\phantomsection\label{papers:papers.api.PublicationResource.Meta}\pysigline{\strong{class }\bfcode{Meta}}
The API serves all {\hyperref[papers:papers.models.Publication]{\code{Publication}}} objects in the database..
\index{object\_class (papers.api.PublicationResource.Meta attribute)}

\begin{fulllineitems}
\phantomsection\label{papers:papers.api.PublicationResource.Meta.object_class}\pysigline{\bfcode{object\_class}}
alias of \code{Publication}

\end{fulllineitems}


\end{fulllineitems}


\end{fulllineitems}



\subsection{Sitemap}
\label{papers:module-papers.sitemap}\label{papers:sitemap}\index{papers.sitemap (module)}
This package controls the sitemap for the {\hyperref[papers:module-papers]{\code{papers}}} app.

This sitemap will be generated at \textbf{/sitemap-papers.xml}
\index{LabPublicationsSitemap (class in papers.sitemap)}

\begin{fulllineitems}
\phantomsection\label{papers:papers.sitemap.LabPublicationsSitemap}\pysigline{\strong{class }\code{papers.sitemap.}\bfcode{LabPublicationsSitemap}}
This sitemap lists all {\hyperref[papers:papers.models.Publication]{\code{Publication}}} objects to be indexed.

It only includes papers from this laboratory (laboratory\_papers=True)
\index{items() (papers.sitemap.LabPublicationsSitemap method)}

\begin{fulllineitems}
\phantomsection\label{papers:papers.sitemap.LabPublicationsSitemap.items}\pysiglinewithargsret{\bfcode{items}}{}{}
Filters {\hyperref[papers:papers.models.Publication]{\code{Publication}}} to show only laboratory papers.

\end{fulllineitems}

\index{lastmod() (papers.sitemap.LabPublicationsSitemap method)}

\begin{fulllineitems}
\phantomsection\label{papers:papers.sitemap.LabPublicationsSitemap.lastmod}\pysiglinewithargsret{\bfcode{lastmod}}{\emph{paper}}{}
lastmod uses the last modification of the paper (not the comments).

\end{fulllineitems}


\end{fulllineitems}



\subsection{Context Processors}
\label{papers:module-papers.context_processors}\label{papers:context-processors}\index{papers.context\_processors (module)}
This file contains context processors to pass api keys to templates as part of the {\hyperref[papers:module-papers]{\code{papers}}} app.

This is needed to properly render the PLOS API requests.
\index{api\_keys() (in module papers.context\_processors)}

\begin{fulllineitems}
\phantomsection\label{papers:papers.context_processors.api_keys}\pysiglinewithargsret{\code{papers.context\_processors.}\bfcode{api\_keys}}{\emph{request}}{}
A context processor to add the a dictionary of api keys to the context.

If no accounts are specified then empty strings should be passed.

\end{fulllineitems}



\chapter{Indices and tables}
\label{index:indices-and-tables}\begin{itemize}
\item {} 
\emph{genindex}

\item {} 
\emph{modindex}

\item {} 
\emph{search}

\end{itemize}


\renewcommand{\indexname}{Python Module Index}
\begin{theindex}
\def\bigletter#1{{\Large\sffamily#1}\nopagebreak\vspace{1mm}}
\bigletter{p}
\item {\texttt{papers}}, \pageref{papers:module-papers}
\item {\texttt{papers.admin}}, \pageref{papers:module-papers.admin}
\item {\texttt{papers.api}}, \pageref{papers:module-papers.api}
\item {\texttt{papers.context\_processors}}, \pageref{papers:module-papers.context_processors}
\item {\texttt{papers.models}}, \pageref{papers:module-papers.models}
\item {\texttt{papers.sitemap}}, \pageref{papers:module-papers.sitemap}
\item {\texttt{papers.tests}}, \pageref{papers:module-papers.tests}
\item {\texttt{papers.urls}}, \pageref{papers:module-papers.urls}
\item {\texttt{papers.views}}, \pageref{papers:module-papers.views}
\end{theindex}

\renewcommand{\indexname}{Index}
\printindex
\end{document}
